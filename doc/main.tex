\documentclass[12pt,a4paper,oneside]{report}

\usepackage{soul}
\usepackage[margin=1in]{geometry}

\usepackage{listings}
\usepackage{color}

\usepackage{float}
\usepackage{animate}

\usepackage{graphicx}
\graphicspath{ {} }

% encode foregin characters
\usepackage[T1]{fontenc}

% hungarian language
\PassOptionsToPackage{defaults=hu-min,classmod=unchanged}{magyar.ldf}
\usepackage{t1enc}
\usepackage[magyar]{babel}
\selectlanguage{magyar}
\usepackage[utf8]{inputenc}

\usepackage[table]{xcolor}
\usepackage{csquotes}
\usepackage{paralist}

\setcounter{tocdepth}{0}


\lstset{language=C++,
	breaklines=true,
	breakatwhitespace=true,
	basicstyle=\ttfamily,
	keywordstyle=\color{blue}\ttfamily,
	stringstyle=\color{red}\ttfamily,
	commentstyle=\color{green}\ttfamily,
	morecomment=[l][\color{magenta}]{\#}
}

\title{%
\vfill
\large \textbf{HALAL}
\vfill
\author{Ráncsik Áron}
\date{\today}
}

\begin{document}
\maketitle
\noindent

\chapter*{Evolúciós algoritmus függvényközelítés}

\section*{Bemeneti változók}

\begin{lstlisting}
GeneticAlgo(
	string input, 
	int population, 
	int mutation, 
	int parents,
	int elitism, 
	double goalFitness
);
GeneticAlgo("FuncAppr1.txt",300,5,2,100,60);
\end{lstlisting}





\section*{Implementáció}
\begin{lstlisting}

#include <algorithm>
#include <iostream>
#include <iomanip>
#include "GeneticAlgo.h"

GeneticAlgo::GeneticAlgo(string input, int population, int mutation, int parents,int elitism, double goalFitness)
{
mMutation =mutation;
mInput =input;
mPopulation=population;
mParents = parents;
mElitism=elitism;
mGoalFitness=goalFitness;
setFunction(mInput);
}
GeneticAlgo::~GeneticAlgo()
{

}
void GeneticAlgo::setFunction(string)
{
funcApproximation = FunctionApproximation();
funcApproximation.loadKnownValuesFromFile(mInput);
}
void GeneticAlgo::Solve()
{
vector<vector<double>> pop = initPopulation();

best = mGetBest(pop);
while (mCalcFitness(best) > mGoalFitness){
vector<vector<double>> newPop = mElite(pop);
while(pop.size()!= newPop.size()) {
vector<vector<double>> parents = mGetParents(pop);
vector<double> chrome = mCrossover(parents);
chrome = mMutate(chrome);
newPop.push_back(chrome);
}
pop = newPop;
best = mGetBest(pop);

for(auto b : best){
cout << b << " ";
}
cout << " fit:  " << mCalcFitness(best)<< endl;
}



}
vector<vector<double>> GeneticAlgo::initPopulation()
{
vector<vector<double>> pop = vector<vector<double>>();
for(int i = 0; i < mPopulation; i++){
vector<double> current = vector<double>();
for(int j = 0; j < 5; j++){
current.push_back(randomUniform(0.0f,7.0f));
}
pop.push_back(current);
}
return pop;
}
vector<double> GeneticAlgo::mGetBest(vector<vector<double>> population)
{
vector<double> max = vector<double>(5,0);
for(auto p : population){
double cur = mCalcFitness(p);
double old = mCalcFitness(max);
if(cur < old){
max = p;
}
}
return max;
}


vector<double> GeneticAlgo::mCrossover(vector<vector<double>> parents)
{
vector<double> child = vector<double>();
child.push_back(parents[1][0]);
child.push_back(parents[1][1]);
child.push_back(parents[0][2]);
child.push_back(parents[0][3]);
child.push_back(parents[0][4]);
return child;
}
vector<double> GeneticAlgo::mMutate(vector<double> chromosome)
{
chromosome[0] *= genMutate();
chromosome[1] *= genMutate();
chromosome[2] *= genMutate();
chromosome[3] *= genMutate();
chromosome[4] *= genMutate();
return chromosome;
}
double GeneticAlgo::genMutate()
{
double min = 1.0f - (mMutation / 100.0f);
double max = 1.0f + (mMutation / 100.0f);
double r = randomUniform(0.0f,1.0f);
return (min+(max-min)*r);
}
double GeneticAlgo::mCalcFitness(vector<double>  chromosome)
{
return funcApproximation.objective(chromosome);
}

vector<vector<double>> GeneticAlgo::mElite(vector<vector<double>> population)
{
sort(population.begin(), population.end(), [this](const vector<double> & a, const vector<double> & b) -> bool
{
return mCalcFitness(a) < mCalcFitness(b);
});

vector<vector<double>> eliteCrew =vector<vector<double>>();
for (int i = 0; i < mElitism; ++i) {
eliteCrew.push_back(population[i]);
}
return eliteCrew;
}
vector<vector<double>> GeneticAlgo::mGetParents(vector<vector<double>> population)
{
sort(population.begin(), population.end(), [this](const vector<double> & a, const vector<double> & b) -> bool
{
return mCalcFitness(a) < mCalcFitness(b);
});

vector<vector<double>> parents =vector<vector<double>>();
for (int i = 0; i < mParents; ++i) {
parents.push_back(population[i]);
}
return parents;
}
\end{lstlisting}

\section*{Mérési eredmények}
  \begin{lstlisting}
0.618766 1.48905 2.05244 4.88465 5.52702  fit:  29777.6
...
...
...
0.570953 1.02322 1.96845 4.68738 5.27573  fit:  25643.6
0.561551 1.05421 1.91757 4.49947 5.01497  fit:  15236.1
0.534307 1.08964 1.85674 4.30761 5.22576  fit:  7469.78
0.516582 1.11512 1.7656 4.09491 5.04041  fit:  1939.85
0.506224 1.15044 1.69475 4.0304 4.94235  fit:  316.893
0.491449 1.10863 1.61454 3.87468 4.78747  fit:  107.631
0.490178 1.12002 1.63977 3.88042 4.55718  fit:  104.855
0.490178 1.12002 1.63977 3.88042 4.55718  fit:  104.855
0.492517 1.07546 1.57269 3.99702 4.52308  fit:  100.782
0.492517 1.07546 1.57269 3.99702 4.52308  fit:  100.782
0.492517 1.07546 1.57269 3.99702 4.52308  fit:  100.782
...
...
...
0.492806 1.14366 1.67846 3.84254 4.5927  fit:  70.1396
0.492806 1.14366 1.67846 3.84254 4.5927  fit:  70.1396
0.492806 1.14366 1.67846 3.84254 4.5927  fit:  70.1396
0.492806 1.14366 1.67846 3.84254 4.5927  fit:  70.1396
0.492806 1.14366 1.67846 3.84254 4.5927  fit:  70.1396
0.496042 1.19049 1.74759 3.80178 4.56571  fit:  52.622

Time elapsed: 131.43039 sec

Process finished with exit code 0
\end{lstlisting}

\section*{Screenshot animáció}
Lejátszáshoz Adobe Acrobat Reader javasolt\newline
\animategraphics[loop,controls,width=10cm]{15}{./images/ga/apngframe}{01}{81}
%\includegraphics[width=\linewidth]{./images/ga/apngframe}


\chapter*{Véletlen optimalizálással utazó ügynök probléma}
\section*{Bemeneti változók}

\begin{lstlisting}
RandomOptAlgo(
	int mMean, 
	int mVariance, 
	double mGoalFitness, 
	const string &filename
)
RandomOptAlgo(0,2,4000,"Towns.txt");
\end{lstlisting}

\section*{Implementáció}
\begin{lstlisting}

#ifndef RANDOMOPTALGO_H
#define RANDOMOPTALGO_H

#include "Random.h"
#include "TravellingSalesman.h"

#include <iostream>

using namespace std;

class RandomOptAlgo
{


public:
	RandomOptAlgo(
		int mMean,
	 	int mVariance, 
	 	double mGoalFitness, 
	 	const string &filename)
		: mMean(mMean), mVariance(mVariance), mGoalFitness(mGoalFitness), filename(filename)
		{
			tsp = TravellingSalesmanProblem();
			tsp.loadTownsFromFile(filename);
		}
		virtual ~RandomOptAlgo()
		{
		
		}
		void Solve(){
			auto route = tsp.getTowns();
			while (CalculateFitness(route) > mGoalFitness){
				cout << "current: fitt: " << CalculateFitness(route)<<"\n";
				
				vector<Town> newRoute = GenerateRoute(route);
				
				
				if(CalculateFitness(newRoute) < CalculateFitness(route)){
					route = newRoute;
					cout << "fitt: "
				 << CalculateFitness(route)<<"\n";
				}
			}
			cout << "fitt: " << CalculateFitness(route)<<"\n";
		}


private:
	int mMean, mVariance;
	double mGoalFitness;
	string filename;
	TravellingSalesmanProblem tsp;
	double CalculateFitness(vector<Town> route){
		tsp.objective(route);
	}

	vector<Town> MixRouteNormalDist(vector<Town> route)
	{
		int n = route.size();
		while (n > 1){
			n--;
			int shift = randomNormal(mMean, mVariance);
			int randomIndex = n + shift;
			if(randomIndex < 0 )
			{
				randomIndex =0;
			}
			if(randomIndex >= route.size())
			{
				randomIndex = route.size()-1;
			}
			Town town = route[randomIndex];
			route[randomIndex] = route[n];
			route[n] = town;
		}
		return route;
	}

vector<Town> GenerateRoute(vector<Town> route)
{
	vector<Town> newRoute = route;
	newRoute = MixRouteNormalDist(newRoute);

	return newRoute;
}
};


#endif //RANDOMOPTALGO_H

\end{lstlisting}
\section*{Mérési eredmények}
\begin{lstlisting}
...
...
...
current: fitt: 4021.08
current: fitt: 4021.08
current: fitt: 4021.08
current: fitt: 4021.08
current: fitt: 4021.08
current: fitt: 4021.08
current: fitt: 4021.08
current: fitt: 4021.08
current: fitt: 4021.08
current: fitt: 4021.08
current: fitt: 4021.08
current: fitt: 4021.08
fitt: 3972.15
fitt: 3972.15

Time elapsed: 4.49370 sec

\end{lstlisting}

\section*{Screenshot animáció}
Lejátszáshoz Adobe Acrobat Reader javasolt\newline
\animategraphics[loop,controls,width=10cm]{15}{./images/ro/apngframe}{037}{161}
%\animategraphics[loop,controls,width=10cm]{15}{./images/ga/apngframe}{01}{81}





\chapter*{Hegymászó algoritmus stochasztikus megvalósítása legkisebb körülíró poligon megoldására}
\section*{Bemeneti változók}

\begin{lstlisting}
HillClimbAlgo(
	int mEpsilon, 
	int mDimension,
	int mMaxCoordinates, 
	int mGoalFitness, 
	const string &filename
)
HillClimbAlgo(10, 3, 400, 1, "Points.txt");
\end{lstlisting}

\section*{Implementáció}
\begin{lstlisting}
#ifndef HILLCLIMBALGO_H
#define HILLCLIMBALGO_H


#include <iostream>
#include "Random.h"
#include "SmallestBoundaryPolygon.h"

using namespace std;

class HillClimbAlgo
{
public:
	HillClimbAlgo(int mEpsilon, int mDimension, int mMaxCoordinates, int mGoalFitness, const string &filename)
	: mEpsilon(mEpsilon), mDimension(mDimension), mMaxCoordinates(mMaxCoordinates), mGoalFitness(mGoalFitness),
	filename(filename)
	{
		sbpp.loadPointsFromFile(filename);
	}
		~HillClimbAlgo(){
		
		}
	void Solve(){
	
		vector<Point> polygon = InitPolygon();
	
		while (CalculateFitness(polygon) > mGoalFitness){
			vector<Point> newPolygon = GenerateRandomPolygon(polygon);
			if(CalculateFitness(newPolygon)
			<
			CalculateFitness(polygon)){
				polygon = newPolygon;
				cout << "new poly: \n";
				cout << ToString(polygon) << " fitt: "<< CalculateFitness(polygon)<<"\n";
			}
		}
		cout << ToString(polygon) << " fitt: "<< CalculateFitness(polygon);
	}
private:

	int mEpsilon, mDimension, mMaxCoordinates, mGoalFitness;
	string filename;
	SmallestBoundaryPolygonProblem sbpp = SmallestBoundaryPolygonProblem();

		double CalculateFitness (vector<Point> polygon){
		double fitt = 0;
		if(sbpp.outerDistanceToBoundary(polygon) != 0){
			fitt = sbpp.objective(polygon);
		}else{
			fitt = 10000000;
		}
		return fitt;
	}

	string ToString (vector<Point> polygon){
		string ret="";
		for(auto p : polygon){
			ret += "(" + to_string(p.x) + "," + to_string(p.y) + ") ";
		}
		return ret;
	}
	
	vector<Point> GenerateRandomPolygon(vector<Point> polygon)
	{
		vector<Point> newPoly = vector<Point>();
			for(auto p : polygon){
			Point newPoint = Point();
			newPoint.x = p.x + randomUniform(-1*mEpsilon, mEpsilon-1);
			newPoint.y = p.y + randomUniform(-1*mEpsilon, mEpsilon-1);
			newPoly.push_back(newPoint);
		}
		return  newPoly;
	}


	vector<Point> InitPolygon(){
		static vector<Point> polygon = vector<Point>();
		for (int i = 0; i < mDimension; ++i) {
			auto p = Point();
			p.x = randomUniform(0,mMaxCoordinates-1);
			p.y = randomUniform(0,mMaxCoordinates-1);
			polygon.push_back(p);
		}
		return polygon;
	}
};


#endif //HILLCLIMBALGO_H

\end{lstlisting}
\section*{Mérési eredmények}
\begin{lstlisting}
/home/aaron/Documents/OE/6felev/halal/AdvAlg-Stud/cmake-build-debug/AdvAlg_Stud 2
new poly: 
(46.000000,158.000000) (134.000000,283.000000) (341.000000,32.000000)  fitt: 478.215
new poly: 
(38.000000,161.000000) (126.000000,282.000000) (342.000000,36.000000)  fitt: 476.988
new poly: 
...
...
...
new poly: 
(197.000000,221.000000) (197.000000,220.000000) (197.000000,221.000000)  fitt: 2
new poly: 
(196.000000,227.000000) (197.000000,228.000000) (197.000000,228.000000)  fitt: 1.41421
new poly: 
(190.000000,224.000000) (190.000000,223.000000) (190.000000,223.000000)  fitt: 1
Time elapsed: 0.69850 sec
\end{lstlisting}

\section*{Screenshot animáció}
Lejátszáshoz Adobe Acrobat Reader javasolt\newline
%\animategraphics[loop,controls,width=10cm]{15}{./images/ga/apngframe}{01}{81}

\end{document}